%
% API Documentation for API Documentation
% Module sakura
%
% Generated by epydoc 3.0.1
% [Sun Dec 15 02:37:54 2013]
%

%%%%%%%%%%%%%%%%%%%%%%%%%%%%%%%%%%%%%%%%%%%%%%%%%%%%%%%%%%%%%%%%%%%%%%%%%%%
%%                          Module Description                           %%
%%%%%%%%%%%%%%%%%%%%%%%%%%%%%%%%%%%%%%%%%%%%%%%%%%%%%%%%%%%%%%%%%%%%%%%%%%%

    \index{sakura \textit{(module)}|(}
\section{Module sakura}

    \label{sakura}
Parses files in the content directory.


%%%%%%%%%%%%%%%%%%%%%%%%%%%%%%%%%%%%%%%%%%%%%%%%%%%%%%%%%%%%%%%%%%%%%%%%%%%
%%                               Functions                               %%
%%%%%%%%%%%%%%%%%%%%%%%%%%%%%%%%%%%%%%%%%%%%%%%%%%%%%%%%%%%%%%%%%%%%%%%%%%%

  \subsection{Functions}

    \label{sakura:iter_tags}
    \index{sakura \textit{(module)}!sakura.iter\_tags \textit{(function)}}

    \vspace{0.5ex}

\hspace{.8\funcindent}\begin{boxedminipage}{\funcwidth}

    \raggedright \textbf{iter\_tags}(\textit{tag}, \textit{document})

\setlength{\parskip}{2ex}
\setlength{\parskip}{1ex}
    \end{boxedminipage}

    \label{sakura:tag_type_exists}
    \index{sakura \textit{(module)}!sakura.tag\_type\_exists \textit{(function)}}

    \vspace{0.5ex}

\hspace{.8\funcindent}\begin{boxedminipage}{\funcwidth}

    \raggedright \textbf{tag\_type\_exists}(\textit{tag}, \textit{document})

    \vspace{-1.5ex}

    \rule{\textwidth}{0.5\fboxrule}
\setlength{\parskip}{2ex}
    Return True if tags with bracket types exist, else false.

\setlength{\parskip}{1ex}
    \end{boxedminipage}

    \label{sakura:minify}
    \index{sakura \textit{(module)}!sakura.minify \textit{(function)}}

    \vspace{0.5ex}

\hspace{.8\funcindent}\begin{boxedminipage}{\funcwidth}

    \raggedright \textbf{minify}(\textit{document\_path}, \textit{document})

    \vspace{-1.5ex}

    \rule{\textwidth}{0.5\fboxrule}
\setlength{\parskip}{2ex}
    Compress CSS and HTML mostly by removing whitespace.

    should be a \%\%func\%\%

\setlength{\parskip}{1ex}
    \end{boxedminipage}

    \label{sakura:include}
    \index{sakura \textit{(module)}!sakura.include \textit{(function)}}

    \vspace{0.5ex}

\hspace{.8\funcindent}\begin{boxedminipage}{\funcwidth}

    \raggedright \textbf{include}(\textit{document}, \textit{include\_directory})

    \vspace{-1.5ex}

    \rule{\textwidth}{0.5\fboxrule}
\setlength{\parskip}{2ex}
    Replaces/parses \%\%inc\%\% calls.

    Example: \%\%inc foo.txt\%\%\%

    document (str) -- document being parsed include\_directory (str) -- 
    directory containing the file to call/include

\setlength{\parskip}{1ex}
    \end{boxedminipage}

    \label{sakura:load_functions}
    \index{sakura \textit{(module)}!sakura.load\_functions \textit{(function)}}

    \vspace{0.5ex}

\hspace{.8\funcindent}\begin{boxedminipage}{\funcwidth}

    \raggedright \textbf{load\_functions}(\textit{public})

    \vspace{-1.5ex}

    \rule{\textwidth}{0.5\fboxrule}
\setlength{\parskip}{2ex}
    Returns a dictionary of "functions" (functions) and their arguments.

    Load functions to evaluate arguments/values pre-defined in "public" 
    (dictionary).

\setlength{\parskip}{1ex}
    \end{boxedminipage}

    \label{sakura:parse}
    \index{sakura \textit{(module)}!sakura.parse \textit{(function)}}

    \vspace{0.5ex}

\hspace{.8\funcindent}\begin{boxedminipage}{\funcwidth}

    \raggedright \textbf{parse}(\textit{document\_path})

    \vspace{-1.5ex}

    \rule{\textwidth}{0.5\fboxrule}
\setlength{\parskip}{2ex}
    Sakura element function, returns string.

    Parse a document in CONTENT; then parse named variables sent to that 
    include, if available.

    document\_path (str) -- path of file being parsed

\setlength{\parskip}{1ex}
    \end{boxedminipage}

    \label{sakura:flush_cache}
    \index{sakura \textit{(module)}!sakura.flush\_cache \textit{(function)}}

    \vspace{0.5ex}

\hspace{.8\funcindent}\begin{boxedminipage}{\funcwidth}

    \raggedright \textbf{flush\_cache}()

    \vspace{-1.5ex}

    \rule{\textwidth}{0.5\fboxrule}
\setlength{\parskip}{2ex}
    Returns True if cache was "flushed."

\setlength{\parskip}{1ex}
    \end{boxedminipage}

    \label{sakura:cache}
    \index{sakura \textit{(module)}!sakura.cache \textit{(function)}}

    \vspace{0.5ex}

\hspace{.8\funcindent}\begin{boxedminipage}{\funcwidth}

    \raggedright \textbf{cache}()

\setlength{\parskip}{2ex}
\setlength{\parskip}{1ex}
    \end{boxedminipage}

    \label{sakura:setup}
    \index{sakura \textit{(module)}!sakura.setup \textit{(function)}}

    \vspace{0.5ex}

\hspace{.8\funcindent}\begin{boxedminipage}{\funcwidth}

    \raggedright \textbf{setup}()

    \vspace{-1.5ex}

    \rule{\textwidth}{0.5\fboxrule}
\setlength{\parskip}{2ex}
    Lazy installer; it sets up the directory structure for Sakura.

\setlength{\parskip}{1ex}
    \end{boxedminipage}

    \label{sakura:backup}
    \index{sakura \textit{(module)}!sakura.backup \textit{(function)}}

    \vspace{0.5ex}

\hspace{.8\funcindent}\begin{boxedminipage}{\funcwidth}

    \raggedright \textbf{backup}()

    \vspace{-1.5ex}

    \rule{\textwidth}{0.5\fboxrule}
\setlength{\parskip}{2ex}
    Zip the config and content directories into a backup/date folder

\setlength{\parskip}{1ex}
    \end{boxedminipage}

    \label{sakura:httpd}
    \index{sakura \textit{(module)}!sakura.httpd \textit{(function)}}

    \vspace{0.5ex}

\hspace{.8\funcindent}\begin{boxedminipage}{\funcwidth}

    \raggedright \textbf{httpd}()

    \vspace{-1.5ex}

    \rule{\textwidth}{0.5\fboxrule}
\setlength{\parskip}{2ex}
    THIS IS A TOY. It is only here so users may test parsed contents before
    making them public.

\setlength{\parskip}{1ex}
    \end{boxedminipage}

    \label{sakura:plugin_remote_install}
    \index{sakura \textit{(module)}!sakura.plugin\_remote\_install \textit{(function)}}

    \vspace{0.5ex}

\hspace{.8\funcindent}\begin{boxedminipage}{\funcwidth}

    \raggedright \textbf{plugin\_remote\_install}(\textit{path})

\setlength{\parskip}{2ex}
\setlength{\parskip}{1ex}
    \end{boxedminipage}

    \label{sakura:zip_file_index}
    \index{sakura \textit{(module)}!sakura.zip\_file\_index \textit{(function)}}

    \vspace{0.5ex}

\hspace{.8\funcindent}\begin{boxedminipage}{\funcwidth}

    \raggedright \textbf{zip\_file\_index}(\textit{zip\_file})

    \vspace{-1.5ex}

    \rule{\textwidth}{0.5\fboxrule}
\setlength{\parskip}{2ex}
    Return a list of paths in zip\_file.

\setlength{\parskip}{1ex}
    \end{boxedminipage}

    \label{sakura:sanity_check}
    \index{sakura \textit{(module)}!sakura.sanity\_check \textit{(function)}}

    \vspace{0.5ex}

\hspace{.8\funcindent}\begin{boxedminipage}{\funcwidth}

    \raggedright \textbf{sanity\_check}(\textit{path})

    \vspace{-1.5ex}

    \rule{\textwidth}{0.5\fboxrule}
\setlength{\parskip}{2ex}
    Assure zip contents adhere to file structure standard.

    path (str) -- the path to the zip to perform a sanity check on

    Yields a two-element tuple, firest element is path, and second is 
    "error."

\setlength{\parskip}{1ex}
    \end{boxedminipage}

    \label{sakura:plugin_check}
    \index{sakura \textit{(module)}!sakura.plugin\_check \textit{(function)}}

    \vspace{0.5ex}

\hspace{.8\funcindent}\begin{boxedminipage}{\funcwidth}

    \raggedright \textbf{plugin\_check}(\textit{path})

    \vspace{-1.5ex}

    \rule{\textwidth}{0.5\fboxrule}
\setlength{\parskip}{2ex}
    Used to check a plugin before installing.

    Assure all files extract to any subdirectories of a sakura system 
    directory, e.g., cgi/, functions/, content/.

    Maybe this should take a zipfile object; should also perform 
    zip\_file.testzip()

\setlength{\parskip}{1ex}
    \end{boxedminipage}

    \label{sakura:file_checksum}
    \index{sakura \textit{(module)}!sakura.file\_checksum \textit{(function)}}

    \vspace{0.5ex}

\hspace{.8\funcindent}\begin{boxedminipage}{\funcwidth}

    \raggedright \textbf{file\_checksum}(\textit{path})

    \vspace{-1.5ex}

    \rule{\textwidth}{0.5\fboxrule}
\setlength{\parskip}{2ex}
    Generate a checksum to compare against later to detect file 
    modification.

\setlength{\parskip}{1ex}
    \end{boxedminipage}

    \label{sakura:plugin_insert}
    \index{sakura \textit{(module)}!sakura.plugin\_insert \textit{(function)}}

    \vspace{0.5ex}

\hspace{.8\funcindent}\begin{boxedminipage}{\funcwidth}

    \raggedright \textbf{plugin\_insert}(\textit{plugin\_path}, *\textit{paths})

    \vspace{-1.5ex}

    \rule{\textwidth}{0.5\fboxrule}
\setlength{\parskip}{2ex}
    Add a series of paths (directories) to a plugin, recursively.

\setlength{\parskip}{1ex}
    \end{boxedminipage}

    \label{sakura:plugin_install}
    \index{sakura \textit{(module)}!sakura.plugin\_install \textit{(function)}}

    \vspace{0.5ex}

\hspace{.8\funcindent}\begin{boxedminipage}{\funcwidth}

    \raggedright \textbf{plugin\_install}(\textit{path}, \textit{update}={\tt False})

    \vspace{-1.5ex}

    \rule{\textwidth}{0.5\fboxrule}
\setlength{\parskip}{2ex}
    Plugin zip-extraction protocol.

\setlength{\parskip}{1ex}
    \end{boxedminipage}

    \label{sakura:plugin_list}
    \index{sakura \textit{(module)}!sakura.plugin\_list \textit{(function)}}

    \vspace{0.5ex}

\hspace{.8\funcindent}\begin{boxedminipage}{\funcwidth}

    \raggedright \textbf{plugin\_list}()

    \vspace{-1.5ex}

    \rule{\textwidth}{0.5\fboxrule}
\setlength{\parskip}{2ex}
    Display installed plugin information.

\setlength{\parskip}{1ex}
    \end{boxedminipage}

    \label{sakura:plugin_delete}
    \index{sakura \textit{(module)}!sakura.plugin\_delete \textit{(function)}}

    \vspace{0.5ex}

\hspace{.8\funcindent}\begin{boxedminipage}{\funcwidth}

    \raggedright \textbf{plugin\_delete}(\textit{name})

    \vspace{-1.5ex}

    \rule{\textwidth}{0.5\fboxrule}
\setlength{\parskip}{2ex}
    Delete file paths associated with plugin.

\setlength{\parskip}{1ex}
    \end{boxedminipage}

    \label{sakura:plugin_error}
    \index{sakura \textit{(module)}!sakura.plugin\_error \textit{(function)}}

    \vspace{0.5ex}

\hspace{.8\funcindent}\begin{boxedminipage}{\funcwidth}

    \raggedright \textbf{plugin\_error}(\textit{plugin})

    \vspace{-1.5ex}

    \rule{\textwidth}{0.5\fboxrule}
\setlength{\parskip}{2ex}
    This sucks. Should have proper exception?.

\setlength{\parskip}{1ex}
    \end{boxedminipage}

    \label{sakura:plugin_info}
    \index{sakura \textit{(module)}!sakura.plugin\_info \textit{(function)}}

    \vspace{0.5ex}

\hspace{.8\funcindent}\begin{boxedminipage}{\funcwidth}

    \raggedright \textbf{plugin\_info}(\textit{plugin})

    \vspace{-1.5ex}

    \rule{\textwidth}{0.5\fboxrule}
\setlength{\parskip}{2ex}
    Display files installed by "plugin."

    Should also print plugin\_meta data.

\setlength{\parskip}{1ex}
    \end{boxedminipage}


%%%%%%%%%%%%%%%%%%%%%%%%%%%%%%%%%%%%%%%%%%%%%%%%%%%%%%%%%%%%%%%%%%%%%%%%%%%
%%                               Variables                               %%
%%%%%%%%%%%%%%%%%%%%%%%%%%%%%%%%%%%%%%%%%%%%%%%%%%%%%%%%%%%%%%%%%%%%%%%%%%%

  \subsection{Variables}

    \vspace{-1cm}
\hspace{\varindent}\begin{longtable}{|p{\varnamewidth}|p{\vardescrwidth}|l}
\cline{1-2}
\cline{1-2} \centering \textbf{Name} & \centering \textbf{Description}& \\
\cline{1-2}
\endhead\cline{1-2}\multicolumn{3}{r}{\small\textit{continued on next page}}\\\endfoot\cline{1-2}
\endlastfoot\raggedright d\-e\-s\-c\-r\-i\-p\-t\-i\-o\-n\- & \raggedright \textbf{Value:} 
{\tt 'Sakura content management system; parses files, then "ca\texttt{...}}&\\
\cline{1-2}
\raggedright f\-u\-n\-c\-t\-i\-o\-n\- & \raggedright \textbf{Value:} 
{\tt argparse.ArgumentParser(description= description, prog= '\texttt{...}}&\\
\cline{1-2}
\raggedright r\-e\-f\-r\-e\-s\-h\-\_\-h\-e\-l\-p\- & \raggedright \textbf{Value:} 
{\tt 'Clear CACHE and reparse CONTENT into CACHE.'}&\\
\cline{1-2}
\raggedright s\-e\-t\-u\-p\-\_\-h\-e\-l\-p\- & \raggedright \textbf{Value:} 
{\tt 'Setup Sakura directories.'}&\\
\cline{1-2}
\raggedright h\-t\-t\-p\-d\-\_\-h\-e\-l\-p\- & \raggedright \textbf{Value:} 
{\tt 'Start HTTPD server.'}&\\
\cline{1-2}
\raggedright i\-n\-s\-t\-a\-l\-l\-\_\-h\-e\-l\-p\- & \raggedright \textbf{Value:} 
{\tt 'Install a plugin (.zip)'}&\\
\cline{1-2}
\raggedright i\-n\-f\-o\-\_\-h\-e\-l\-p\- & \raggedright \textbf{Value:} 
{\tt 'Display files belonging to a plugin.'}&\\
\cline{1-2}
\raggedright u\-p\-d\-a\-t\-e\-\_\-h\-e\-l\-p\- & \raggedright \textbf{Value:} 
{\tt 'Update a plugin by name.'}&\\
\cline{1-2}
\raggedright i\-n\-s\-e\-r\-t\-\_\-h\-e\-l\-p\- & \raggedright \textbf{Value:} 
{\tt 'Add a series of paths to a plugin, recursively.'}&\\
\cline{1-2}
\raggedright c\-h\-e\-c\-k\-\_\-h\-e\-l\-p\- & \raggedright \textbf{Value:} 
{\tt 'Check a plugin before you install it!'}&\\
\cline{1-2}
\raggedright d\-e\-l\-e\-t\-e\-\_\-h\-e\-l\-p\- & \raggedright \textbf{Value:} 
{\tt 'Delete a plugin (by name)'}&\\
\cline{1-2}
\raggedright l\-i\-s\-t\-\_\-h\-e\-l\-p\- & \raggedright \textbf{Value:} 
{\tt 'List installed plugins'}&\\
\cline{1-2}
\raggedright b\-a\-c\-k\-u\-p\-\_\-h\-e\-l\-p\- & \raggedright \textbf{Value:} 
{\tt 'Backup defined Sakura directories.'}&\\
\cline{1-2}
\raggedright a\-r\-g\-s\- & \raggedright \textbf{Value:} 
{\tt function.parse\_args()}&\\
\cline{1-2}
\end{longtable}


%%%%%%%%%%%%%%%%%%%%%%%%%%%%%%%%%%%%%%%%%%%%%%%%%%%%%%%%%%%%%%%%%%%%%%%%%%%
%%                           Class Description                           %%
%%%%%%%%%%%%%%%%%%%%%%%%%%%%%%%%%%%%%%%%%%%%%%%%%%%%%%%%%%%%%%%%%%%%%%%%%%%

    \index{sakura \textit{(module)}!sakura.ThreadingCGIServer \textit{(class)}|(}
\subsection{Class ThreadingCGIServer}

    \label{sakura:ThreadingCGIServer}
\begin{tabular}{cccccccccc}
% Line for SocketServer.ThreadingMixIn, linespec=[False]
\multicolumn{6}{r}{\settowidth{\BCL}{SocketServer.ThreadingMixIn}\multirow{2}{\BCL}{SocketServer.ThreadingMixIn}}
&&
  \\\cline{7-7}
  &&&&&&\multicolumn{1}{c|}{}
&&
  \\
% Line for SocketServer.BaseServer, linespec=[False, False, True]
\multicolumn{2}{r}{\settowidth{\BCL}{SocketServer.BaseServer}\multirow{2}{\BCL}{SocketServer.BaseServer}}
&&
&&
&&\multicolumn{1}{|c}{}
  \\\cline{3-3}
  &&\multicolumn{1}{c|}{}
&&
&&
&\multicolumn{1}{|c}{}&
  \\
% Line for SocketServer.TCPServer, linespec=[False, True]
\multicolumn{4}{r}{\settowidth{\BCL}{SocketServer.TCPServer}\multirow{2}{\BCL}{SocketServer.TCPServer}}
&&
&&\multicolumn{1}{|c}{}
  \\\cline{5-5}
  &&&&\multicolumn{1}{c|}{}
&&
&\multicolumn{1}{|c}{}&
  \\
% Line for BaseHTTPServer.HTTPServer, linespec=[True]
\multicolumn{6}{r}{\settowidth{\BCL}{BaseHTTPServer.HTTPServer}\multirow{2}{\BCL}{BaseHTTPServer.HTTPServer}}
&&\multicolumn{1}{|c}{}
  \\\cline{7-7}
  &&&&&&\multicolumn{1}{c|}{}
&\multicolumn{1}{|c}{}&
  \\
&&&&&&\multicolumn{2}{l}{\textbf{sakura.ThreadingCGIServer}}
\end{tabular}


%%%%%%%%%%%%%%%%%%%%%%%%%%%%%%%%%%%%%%%%%%%%%%%%%%%%%%%%%%%%%%%%%%%%%%%%%%%
%%                                Methods                                %%
%%%%%%%%%%%%%%%%%%%%%%%%%%%%%%%%%%%%%%%%%%%%%%%%%%%%%%%%%%%%%%%%%%%%%%%%%%%

  \subsubsection{Methods}


\large{\textbf{\textit{Inherited from SocketServer.ThreadingMixIn}}}

\begin{quote}
process\_request(), process\_request\_thread()
\end{quote}

\large{\textbf{\textit{Inherited from BaseHTTPServer.HTTPServer}}}

\begin{quote}
server\_bind()
\end{quote}

\large{\textbf{\textit{Inherited from SocketServer.TCPServer}}}

\begin{quote}
\_\_init\_\_(), close\_request(), fileno(), get\_request(), server\_activate(), server\_close(), shutdown\_request()
\end{quote}

\large{\textbf{\textit{Inherited from SocketServer.BaseServer}}}

\begin{quote}
finish\_request(), handle\_error(), handle\_request(), handle\_timeout(), serve\_forever(), shutdown(), verify\_request()
\end{quote}

%%%%%%%%%%%%%%%%%%%%%%%%%%%%%%%%%%%%%%%%%%%%%%%%%%%%%%%%%%%%%%%%%%%%%%%%%%%
%%                            Class Variables                            %%
%%%%%%%%%%%%%%%%%%%%%%%%%%%%%%%%%%%%%%%%%%%%%%%%%%%%%%%%%%%%%%%%%%%%%%%%%%%

  \subsubsection{Class Variables}

    \vspace{-1cm}
\hspace{\varindent}\begin{longtable}{|p{\varnamewidth}|p{\vardescrwidth}|l}
\cline{1-2}
\cline{1-2} \centering \textbf{Name} & \centering \textbf{Description}& \\
\cline{1-2}
\endhead\cline{1-2}\multicolumn{3}{r}{\small\textit{continued on next page}}\\\endfoot\cline{1-2}
\endlastfoot\multicolumn{2}{|l|}{\textit{Inherited from SocketServer.ThreadingMixIn}}\\
\multicolumn{2}{|p{\varwidth}|}{\raggedright daemon\_threads}\\
\cline{1-2}
\multicolumn{2}{|l|}{\textit{Inherited from BaseHTTPServer.HTTPServer}}\\
\multicolumn{2}{|p{\varwidth}|}{\raggedright allow\_reuse\_address}\\
\cline{1-2}
\multicolumn{2}{|l|}{\textit{Inherited from SocketServer.TCPServer}}\\
\multicolumn{2}{|p{\varwidth}|}{\raggedright address\_family, request\_queue\_size, socket\_type}\\
\cline{1-2}
\multicolumn{2}{|l|}{\textit{Inherited from SocketServer.BaseServer}}\\
\multicolumn{2}{|p{\varwidth}|}{\raggedright timeout}\\
\cline{1-2}
\end{longtable}

    \index{sakura \textit{(module)}!sakura.ThreadingCGIServer \textit{(class)}|)}
    \index{sakura \textit{(module)}|)}
